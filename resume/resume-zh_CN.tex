% !TEX TS-program = xelatex
% !TEX encoding = UTF-8 Unicode
% !Mode:: "TeX:UTF-8"

\documentclass{resume}
\usepackage{zh_CN-Adobefonts_external} % Simplified Chinese Support using external fonts (./fonts/zh_CN-Adobe/)
%\usepackage{zh_CN-Adobefonts_internal} % Simplified Chinese Support using system fonts
\usepackage{linespacing_fix} % disable extra space before next section
\usepackage{cite}

\begin{document}
\pagenumbering{gobble} % suppress displaying page number

\name{王正一}

% {E-mail}{mobilephone}{homepage}
% be careful of _ in emaill address
\contactInfo{wzyll1314@gmail.com}{(+86) 138-1147-5902}{}
% {E-mail}{mobilephone}
% keep the last empty braces!
%\contactInfo{xxx@yuanbin.me}{(+86) 131-221-87xxx}{}
 
\section{\faGraduationCap\  教育背景}
\datedsubsection{\textbf{中国传媒大学}}{2011.07 -- 2014.06}
\textit{工学硕士}\ 计算机应用技术
\datedsubsection{\textbf{青岛大学}}{2007.09 -- 2011.06}
\textit{工学学士}\ 网络工程


\section{\faUsers\ 工作经历}
\datedsubsection{\textbf{阿里巴巴} OS事业群, 北京, 研发工程师}{2014年3月 -- 至今}
\role{\textbf{智能硬件-支付手表项目}} {2015/07 -- 至今}
负责支付手表整个应用层模块开发,具体包括:
\begin{itemize}
  \item 网络框架:根据Volley设计并实现网络框架,实现网络并发控制,请求结果回调以及网络请求硬盘缓存功能,并根据业务需求自定制网络请求,支持网络字节流解析成业务需要的数据对象.
  \item 天气应用: 自定义数据库结构存储地理位置信息,合理实现对象序列化实现天气信息硬盘缓存,接入高德定位并通过Service定时拉取最新天气数据,实现Widget并通过Service操作RemoteView对Widget定时更新.
  \item 秒表、闹钟、计时器:为了适配圆形手表,设计并编写大量自定义控件,用来实现表盘、刻度等UI细节.独立设计并实现刻度替换算法,可以在表盘里合理的显示分钟进制刻度和小时进制刻度.并且提供ContentProvider和AIDL通信,便于其他进程获取闹钟、秒表等数据信息. 
  \item 设置应用:在Android原生设置应用的基础上,根据手表UI重新实现设置应用,包括亮度调节、声音调节、WIFI模块等.
  \item 密码锁:实现第三方密码锁功能,并且深入理解Keyguard锁屏机制,改写Framework的Keyguard代码,替换原生锁屏系统.
  \item 自定义控件:会抽象出通用自定义控件,打成jar包或集成到系统以Library Project形式提供给团队同学使用.
\end{itemize}

\role{\textbf{YunOS Rom适配}}{2014/11 -- 2015/06}
\begin{onehalfspacing}
负责MTK机型的YunOS Rom适配.主要是通过编译和拼包的方式将YunOS的Rom和MTK具体机型的Rom进行拼装移植.具体包括:
\begin{itemize}
  \item 写Android.mk单独修改和编译模块,掌握Android Rom编译体系,并对boot.img和system.img进行解包和重新打包.
  \item 从abd log分析分析系统起机遇到的问题,并能通过修改init模块,分析zygote和system server源码来解决相应的问题.
  \item 编写Edify刷机脚本,通过Fastboot和Recovery对机型进行移植测试.并编译shell脚本,将重复过程脚本化.
  \item 熟悉Android Framework层jar包的作用和JNI层大部分动态库的作用,便于遇到问题时替换相应的so库.
\end{itemize}
\end{onehalfspacing}

\role{\textbf{YunOS移动端论坛}}{2014/07 -- 2014/11}
\begin{onehalfspacing}
负责YunOS论坛移动端研发.这也是我刚接触Android就短期内独立研发并多次迭代上线的一款YunOS官方应用,目前在官方应用市场下载量超过百万.具体技术包括:
\begin{itemize}
  \item 设计应用的实现架构和目录结构.
  \item 设计并封装了HTTP网络请求,提供队列机制对网络请求进行并发处理.
  \item 设计了ViewPager+Fragment的UI架构,并且自定义控件实现ViewPageIndicator,可跟随ViewPager进行滑动和颜色的改变,并能响应click事件.
  \item 封装并使用WebView,并定制JavaScript接口来进行特殊请求处理.
\end{itemize}
\end{onehalfspacing}

\datedsubsection{\textbf{北京灵创众和科技有限公司}, 服务端研发工程师}{2011/11 - 2013/09}
\role{用户消息系统}{2012/05 - 2012/09}
\begin{onehalfspacing}
构建好联系用户的消息存储系统. 基于Redis数据库设计并实现消息实时读取算法, 提供用户消息的实时存储, 读取, 删除等功能, 支撑了消息通知, 名片分享等业务功能的实现.
\end{onehalfspacing}

\role{移动黄页搜索系统}{2011/12 - 2012/04}
\begin{onehalfspacing}
设计并构建基于 Coreseek+mysql+redis 的本地黄页数据移动搜索引擎, 提供支持 JSON的 API 查询接口, 支持百万数据的索引, 中文分词, 准实时索引更新, 数据筛选.
\end{onehalfspacing}

\role{服务器运维}{2011/11 - 2013/09}
\begin{onehalfspacing}
搭建服务器端的 LNMP 基础环境, 提出并实现基于bash脚本+rsync的增量代码部署方案,负责线上服务器安全加固, mysql、redis 数据备份和恢复等运维工作, 期间完成过服务器端 apache2+mod\_php 到 nginx+fastcgi+fpm 的架构迁移.
\end{onehalfspacing}

% Reference Test
%\datedsubsection{\textbf{Paper Title\cite{zaharia2012resilient}}}{May. 2015}
%An xxx optimized for xxx\cite{verma2015large}
%\begin{itemize}
%  \item main contribution
%\end{itemize}

\section{\faCogs\ IT 技能}
% increase linespacing [parsep=0.5ex]
\begin{itemize}[parsep=0.5ex]
  \item Android应用层: 熟练掌握Android UI实现, 自定义View, 多线程通信, 进程间通信, HTTP/HTTPS/Socket网络通信, 能够实现自定义网络请求框架, 并对Android源码有一定的研究.
  \item Android Framework层: 熟悉Android体系架构和JNI开发, 对AOSP ROM适配有一定的开发经验.
  \item 前端: 主导过Vue框架开源项目实践.
  \item 编程语言: 熟练使用JAVA, PHP, C, Shell, JavaScript, 了解HTML, CSS.
  \item 数据库: 熟悉SQLite, MySQL, Redis.
  \item 运维: 熟悉Linux和Linux系统运维, 能够实现基于Nginx的负载均衡web服务器.
  \item 基础: 熟悉并掌握常用算法, 数据结构和设计模式.
  \item 英语: CET-6, Google无障碍搜索阅读英文资料.
\end{itemize}

\section{\faInfo\ 其他}
% increase linespacing [parsep=0.5ex]
\begin{itemize}[parsep=0.5ex]
  \item Blog: http://csdn.net/wzy\_1988
  \item GitHub: https://github.com/wangzhengyi
\end{itemize}

%% Reference
%\newpage
%\bibliographystyle{IEEETran}
%\bibliography{mycite}
\end{document}
